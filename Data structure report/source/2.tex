\documentclass[a4paper,12pt]{extreport}

\usepackage[english,russian]{babel}
\usepackage[T2A]{fontenc}
\usepackage[utf8]{inputenc}

\usepackage{indentfirst}
\setlength\parindent{5ex}

\usepackage{geometry}
\geometry{left=3cm}
\geometry{right=1.5cm}
\geometry{top=2.4cm}
\geometry{bottom=2.4cm}

\usepackage{minted}

\begin{document}

\begin{center}
    Введение
\end{center}

Дан список группы в виде: Фамилия, Пол, Результаты сессии. Определить лидеров среди мужчин и женщин по успеваемости и их среди балл (у мужчин и жещин отдельн).

\textbf{Цель работы:} Одна и та же информация из входного файла вводится по-разному в оперативную память с целью освооения работы с различными структурами данных - задание выполняется в виде 5 отдельных программных проектов, где необходимо использовать разные способы обработки данных, то есть: 

\begin{itemize}
    \item массивы строк;
    \item массивы символов;
    \item внутренние процедуры;
    \item массивы структур или структур массивов;
    \item файлы записей;
    \item модули;
    \item хвостовая рекурсия;
    \item днонаправленные списки заранее неизвестной длины;
    \item регулярное программирование.
\end{itemize}

Реализовать задание с использованием массивов строк.

Реализовать задание с использованием массивов символов.

Реализовать задание с использованием массивов строк или строк массивов.

Реализовать задание с использованием динамической структуры данных.

\newpage

\chapter{}

\renewcommand{\labelenumii}{\arabic{enumi}.\arabic{enumii}.}

Данные на вход

\begin{minted}{text}
    Дудиков         Д. Р. М 43453 0.00
    Тихонов         Л. П. М 55353 0.00
    Степин          К. Д. М 55445 0.00
    Садовникова     П. О. Ж 43555 0.00
    Воробъёва       Е. Р. Ж 44553 0.00
\end{minted}

Данные на выход

 \begin{minted}{text}
    Исходный список:
    Дудиков         Д. Р. М 43453 0.00
    Тихонов         Л. П. М 55353 0.00
    Степин          К. Д. М 55445 0.00
    Садовникова     П. О. Ж 43555 0.00
    Воробъёва       Е. Р. Ж 44553 0.00
    
    Лучшая успеваемость среди юношей:
    Степин          К. Д. М 55445 4.60
    
    Лучшая успеваемость среди девочек:
    Садовникова     П. О. Ж 43555 4.40
    
    Средний балл среди юношей:  4.20
    
    Средний балл среди девочек:  4.30
\end{minted}

\begin{enumerate}
    \item Реализация задания с использованием массивов строк
    
Строковые массивы хранят части текста и обеспечивают набор функций для работы с текстом как данные. Можно индексировать в, измениться и конкатенировать массивы строк, как вы можете с массивами любого другого типа. Также можно получить доступ к символам в строке и добавить текст к строкам.    
    
         \begin{minted}{fortran}
integer, parameter    :: STUD_AMOUNT = 5, SURNAME_LEN = 15, &
INITIALS_LEN = 5, MARKS_AMOUNT = 5
character(kind=CH_), parameter :: MALE = Char(1052, CH_) 
character(kind=CH_), parameter :: FEMALE = Char(1046, CH_) 
integer : In, Out, IO, i
character(SURNAME_LEN, kind=CH_):: Surnames(STUD_AMOUNT) = ""
character(SURNAME_LEN, kind=CH_), allocatable:: Boys_Surnames(:), &
                                                Girls_Surnames(:)
character(INITIALS_LEN, kind=CH_):: Initials(STUD_AMOUNT) = ""
character(INITIALS_LEN, kind=CH_), allocatable  :: Boys_Initials(:), &
                                                    Girls_Initials(:)

character(kind=CH_):: Gender(STUD_AMOUNT) = ""

open (file=input_file, encoding=E_, newunit=In)
        format = '(3(a, 1x), ' // MARKS_AMOUNT // 'i1, f5.2)' 
        read (In, format, iostat=IO) (Surnames(i), Initials(i), &
        Gender(i), Marks(i, :), Aver_Marks(i), i = 1, STUD_AMOUNT)
    close (In)

do concurrent (i = 1:Boys_Amount)
    Boys_Surnames(i)  = Surnames(Boys_Pos(i))
    Boys_Initials(i)  = Initials(Boys_Pos(i))
    Boys_Marks(i, :)  = Marks(Boys_Pos(i), :)
end do
   
do concurrent (i = 1:Girls_Amount)
    Girls_Surnames(i)  = Surnames(Girls_Pos(i))
    Girls_Initials(i)  = Initials(Girls_Pos(i))
    Girls_Marks(i, :)  = Marks(Girls_Pos(i), :)
end do
\end{minted}

\item Реализация задания с использованием массивов символов
    
Для получения регулярного доступа к оперативной памяти и сплощным данным в памяти, нужно назначить индексы в двумерном массиве символов \textbf{(Names(LEN, AMOUNT)}, тем самым считывая данные из списка по столбцам на не по строкам.
\begin{minted}{fortran}
integer, parameter  :: STUD_AMOUNT = 5, SURNAME_LEN = 15, &
INITIALS_LEN = 5, MARKS_AMOUNT = 5
character(kind=CH_), parameter:: MALE = Char(1052, CH_)
character(kind=CH_),parameter:: FEMALE = Char(1046, CH_)
character(kind=CH_):: Surnames(SURNAME_LEN, STUD_AMOUNT)  = "", &
                           Initials(INITIALS_LEN, STUD_AMOUNT) = "", &                    
                           Genders(STUD_AMOUNT) = ""

character(kind=CH_), allocatable:: Boys_Surnames(:, :), Girls_Surnames(:, :)
character(kind=CH_), allocatable:: Boys_Initials(:, :), Girls_Initials(:, :)

integer:: Marks(MARKS_AMOUNT, STUD_AMOUNT) = 0, i = 0
integer, allocatable:: Boys_Marks(:, :), Girls_Marks(:, :)


subroutine Read_class_list(Input_File, Surnames, Initials, Genders, &
                            Marks, Aver_Marks)
        character(*)         Input_File
        character(kind=CH_)  Surnames(:, :), Initials(:, :), Genders(:)
        integer              Marks(:, :)
        real(R_)             Aver_Marks(:)
        intent (in)          Input_File
        intent (out)         Surnames, Initials, Genders, Marks, Aver_Marks

        integer In, IO, i
        character(:), allocatable  :: format
      
        open (file=Input_File, encoding=E_, newunit=In)
            format = '(' // SURNAME_LEN // 'a1, 1x, ' // INITIALS_LEN // &
            'a1, 1x, a, 1x, ' // MARKS_AMOUNT // 'i1, f5.2)'
            read (In, format, iostat=IO) (Surnames(:, i), Initials(:, i), &
            Genders(i), Marks(:, i), Aver_Marks(i), &
            i = 1, STUD_AMOUNT)
            call Handle_IO_status(IO, "reading class list")
        close (In)
    end subroutine Read_class_list


allocate (Gender_Surnames(SURNAME_LEN, Gender_Amount), &
Gender_Initials(INITIALS_LEN, Gender_Amount), Gender_Marks(MARKS_AMOUNT, &
                                        Gender_Amount))

do concurrent (i = 1:Gender_Amount)
    Gender_Surnames(:, i)  = Surnames(:, Gender_Pos(i))
    Gender_Initials(:, i)  = Initials(:, Gender_Pos(i))
    Gender_Marks(:, i)  = Marks(:, Gender_Pos(i))
end do
\end{minted}
         
\item Реализация задания с использованием массив структур
\begin{minted}{FORTRAN}
function Read_class_list(File_Data) result(Group)
type(student)                 Group(STUD_AMOUNT)
character(*), intent(in)   :: File_Data

integer In, IO, recl

recl = ((SURNAME_LEN + INITIALS_LEN + 1)*CH_ + &
    MARKS_AMOUNT*I_ + R_) * STUD_AMOUNT
open (file=File_Data, form='unformatted', newunit=In, &
    access='direct', recl=recl)
    read (In, iostat=IO, rec=1) Group
    call Handle_IO_status(IO, "reading unformatted class list")
close (In)
end function Read_class_list

Boys  = Pack(Group, Group%Sex == MALE)
Girls = Pack(Group, Group%Sex == FEMALE)

do concurrent (i = 1:Size(Boys))
Boys(i)%Aver_mark = Real(Sum(Boys(i)%Marks), R_) / MARKS_AMOUNT
end do

do concurrent (i = 1:Size(Girls))
Girls(i)%Aver_mark = Real(Sum(Girls(i)%Marks), R_) / MARKS_AMOUNT
end do
\end{minted}

    \item Реализация задания с использованием структур массивов
\begin{minted}{FORTRAN}
function Read_class_list(Data_File) result(Group)
type(student)                 Group
character(*), intent(in)   :: Data_File

integer In, IO

open (file=Data_File, form='unformatted', newunit=In, access='stream')
    read (In, iostat=IO) Group
    call Handle_IO_status(IO, "reading unformatted class list")
close (In)
end function Read_class_list

call Output_class_list(output_file, Group, INDEXES,  "Исходный список:", "rewind")

do concurrent (i = 1:Size(Group%Aver_mark))
Group%Aver_mark(i) = Real(Sum(Group%Marks(:,i)), R_) / MARKS_AMOUNT
end do
\end{minted}
    \item Реализация задания с использованием динамического списка
\begin{minted}{FORTRAN}
function Read_class_list(Input_File) result(Class_List)
    type(student), pointer     :: Class_List
    character(*), intent(in)   :: Input_File
    integer  In

    open (file=Input_File, encoding=E_, newunit=In)
        Class_List => Read_student(In)        
    close (In)
end function Read_class_list

recursive function Read_student(In) result(Stud)
    type(student), pointer  :: Stud
    integer, intent(in)     :: In
    integer  IO
    character(:), allocatable  :: format

    allocate (Stud)
    
    format = '(3(a, 1x), ' // MARKS_AMOUNT // 'i1, f5.2)'
    read (In, format, iostat=IO) stud%Surname, &
        stud%Initials, stud%Sex, stud%Marks, stud%Aver_Mark
    call Handle_IO_status(IO, "reading line from file")
    if (IO == 0) then
        Stud%next => Read_student(In)
    else
        deallocate (Stud)
        nullify (Stud)
    end if
end function Read_student

Group_List => Read_class_list(input_file)

if (Associated(Group_List)) then
  call Output_class_list(output_file, Group_List, 0.0, .TRUE., .TRUE., &
        "Исходный список:", "rewind")
  call Max_and_sum (Group_List, Boys_Max_Value, Girls_Max_Value, &
    Sum_Boys_Amount, Sum_Girls_Amount, Number_of_Boys, Number_of_Girls)
  Boys_Average_Marks = Sum_Boys_Amount / Number_of_Boys
  Girls_Average_Marks = Sum_Girls_Amount / Number_of_Girls
end if
\end{minted}

\item Основные операторы обработки данных.
    \begin{itemize}
         \item {\textbf{Open}}
        \item {\textbf{Write}}
        \item {\textbf{Do concurrent}}
        \item {\textbf{If, Elseif, Else}}
        \item {\textbf{Type}}
    \end{itemize}
\item При использование стандартных функций в fortran код всегда будет векторизоваться.  Данные будут векторизоваться, если однопоточные приложения, выполняющие одну операцию в каждый момент времени, модифицируется для выполнения нескольких однотипных опрераций одноверменно. 

\item При выборе использования массивов структур или структур массивов, нужно исходить из самого задание, что вам нужно получить и какие данные вам нужно обработать. При использование массива структур, а не структур массивов, когда алгоритм подразумевает работу с данными конкретной структуры, а не со множеством всех элементов вообще. 

\item Динамическая структура.
    \begin{enumerate}
    \item Ссылки и адресаты
        \begin{enumerate}
            \item Ссылка - переменная, связанная с другой переменной, называемой адресатом. Обращении к ссылке будет происходить обращение к адресату и наоборот.
            \item Ссылки позволяют создавать динамические структуры данных - списки, деревья, очереди.
            \item Все изменения происходящие с адресатом, дублируется в ссылке.
            \item Объект называется недостежимым если на него нельзя ссылаться.
        \end{enumerate}
    \end{enumerate}
    \item Формирование двоичного файла. Двоичный файл создаётся для производительности, так же конфигурационный файл формируется в двоичном файле. Читобы работать с двоичными данными, нужно знать формат записи.
\begin{minted}{FORTRAN}
subroutine Create_data_file(Input_File, Data_File)
    character(*), intent(in)   :: Input_File, data_file
  
    type(student)              :: stud
    integer                    :: In, Out, IO, i, recl
    character(:), allocatable  :: format
  
    open (file=Input_File, encoding=E_, newunit=In)
        recl = (SURNAME_LEN + INITIALS_LEN + 1)*CH_ + MARKS_AMOUNT*I_ + R_
        open (file=Data_File, form='unformatted', &
        newunit=Out, access='direct', recl=recl)
            format = '(3(a, 1x), ' // MARKS_AMOUNT // 'i1, f5.2)'
            do i = 1, STUD_AMOUNT
                read (In, format, iostat=IO) stud
                call Handle_IO_status(IO, "reading &
                formatted class list, line " // i)
                write (Out, iostat=IO, rec=i) stud
                call Handle_IO_status(IO, "creating unformatted &
                file with class list, record " // i)
            end do
        close (In)
    close (Out)
end subroutine Create_data_file

character(:), allocatable   :: input_file, output_file, data_file

input_file  = "../data/class.txt"
output_file = "output.txt"
data_file   = "class.dat"
   
call Create_data_file(input_file, data_file)

Group = Read_class_list(data_file)
\end{minted}
\end{enumerate}

\chapter{}

\begin{enumerate}
     
\item Сравнение реализаций
            \begin{enumerate}
                \item Сравнительная таблица массива структур, структур массивов, динамический список
                
                \begin{tabular}{ | l | l | l | l | }
                \hline
                    реализация & 1.3 & 1.4 & 1.5 \\ \hline
                    сплошные данные & - & + & -  \\
                    регулярный доступ & + & + & + \\
                    векторизация & - & + & - \\
                    понециальная векторизация & - & + & - \\
                \hline
                \end{tabular}
                \item При выборе структур массиво данные в оперативной памяти будут сплошными, так же будет регулярный доступ к памяти и следовательно возможна реализация векторизации. 
            \end{enumerate}
            \item Вывод
            
            При выполнении лабораторной работы 1, были использованы и реализованы несколько способов реализации структуры данных. Для решения задачи, было выбрана структура массивов так как при её испольовании данные в оперативной памяти будут сплошными, так же будет регулярный доступ к памяти и следовательно возможна реализация векторизации, тем самым мы увеличиваем производительность при обработке данных.
\end{enumerate}
\end{document}
